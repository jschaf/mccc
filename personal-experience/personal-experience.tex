\documentclass[]{article}
\usepackage{lmodern}
\usepackage{amssymb,amsmath}
\usepackage{ifxetex,ifluatex}
\usepackage{fixltx2e} % provides \textsubscript
\ifnum 0\ifxetex 1\fi\ifluatex 1\fi=0 % if pdftex
  \usepackage[T1]{fontenc}
  \usepackage[utf8]{inputenc}
\else % if luatex or xelatex
  \ifxetex
    \usepackage{mathspec}
    \usepackage{xltxtra,xunicode}
  \else
    \usepackage{fontspec}
  \fi
  \defaultfontfeatures{Mapping=tex-text,Scale=MatchLowercase}
  \newcommand{\euro}{€}
\fi
% use upquote if available, for straight quotes in verbatim environments
\IfFileExists{upquote.sty}{\usepackage{upquote}}{}
% use microtype if available
\IfFileExists{microtype.sty}{\usepackage{microtype}}{}
\ifxetex
  \usepackage[setpagesize=false, % page size defined by xetex
              unicode=false, % unicode breaks when used with xetex
              xetex]{hyperref}
\else
  \usepackage[unicode=true]{hyperref}
\fi
\hypersetup{breaklinks=true,
            bookmarks=true,
            pdfauthor={},
            pdftitle={Joe Schafer's Personal Battle Experience},
            colorlinks=true,
            citecolor=blue,
            urlcolor=blue,
            linkcolor=magenta,
            pdfborder={0 0 0}}
\urlstyle{same}  % don't use monospace font for urls
\setlength{\parindent}{0pt}
\setlength{\parskip}{6pt plus 2pt minus 1pt}
\setlength{\emergencystretch}{3em}  % prevent overfull lines
\setcounter{secnumdepth}{0}

\title{Joe Schafer's Personal Battle Experience}

\begin{document}
\maketitle

It was a dark and stormy night. All across the land nervous paratroopers
filed onto Bagram Air Field. Bagram Air Field, known commonly as BAF was
a hustle of activity all through the night. We moved out to Camp
Montrond, the headquarters of Speical Operations Task Force - East,
(SOTF-E). SOTF-E was regionally aligned with RC-East and was where we
were to conduct combat operations for the next nine months.

I was the platoon leader of Anti-tank Platoon 4 (AT4). Most of our
battalion was dispersed by squads and sections to the Special Forces
teams, Operational Detachment Alpha (ODA) across the whole of
Afghanistan. Delta company, my company, was instead broken down by
Platoon and tasked to act as the forward logistics element. We would
bring supplies to ODAs in our region.

We inprocessed for a couple of weeks at BAF, and did our left seat/right
seat rides with the existing FLE at Camp Montrond. We eventually moved
down to Gardez and took over a small portion of FOB Lightning and
appropriatley named it Camp Destroyer. Our call sign was strong arm,
chosen because the sum of the commander's and the first sergeant's bicep
was ?45in.

A week later, in July 2011, we conducted our first solo mission. We were
conducting a route reconnaissance. We were following an 11 truck route
clearance patrol up to COP Herrera. Afghan roads are like the
highlander, there can be only one. There was exactly one way to get to
each COP or VSP we would travel to. RCP placed named areas of interest
on portions of the road with more risk of an IED. While traveling
through an NAI. The RCP would send two soldiers ahead of the convoy to
check culverts for IEDs. The rationale was, that the Taliban would not
``waste'' an IED on only a soldier. Once we passed through an NAI,
everyone would mount up and we would travel at our normal convoy speed.

Unfortunately, this is when I learned the mantra, just because RCP
traveled the route, it doesn't mean it is clear. About two hundred
meters outside the NAI, the Taliban detonated a command wire IED on the
last truck in the convoy. Because we were trailing RCP, it was my PSGs
truck. The blast blew the truck up and off the road, into a shallow
creek. No one was injured apart from some minor concussions. This is
when I really learned to listen to my platoon sergeant. While we
conducted our relief in place with orignal FLE at Camp Montrond, the SF
soldiers would have a trunk monkey stand during the entire ride. My PSG
thought this was insane in an area known more for large IEDs than for
direct fire ambushes, so he told all his paratroopers to sit down. This
foresight undoubtedly saved CPL Copeland's life.

\textless{}picture\textgreater{}

We experimented with different TTPs for several convoys to reduce the
chance of getting blown up. In my experience, RCP was largely
ineffective for finding IEDs. The taliban favored command-wire IEDs due
to the overwhelming precense of jammers we carried. Fortunately, they
were lazy and didn't like night shifts. We transitioned to operating
solely at night. We didn't get blown up once at night.

In addition to the IEDs, the Taliban would often conduct ambushes on our
convoy. The ambushes were always ineffective and never paired with an
IED for a complex ambush. We would often take AK-47 fire at our RG-33
which didn't do anything. The only real danger was an RPG which were
rarely used.

Our choice of weapon system influenced how often we were attacked. The
enemy was familiar with the capabilities of our various weapons systems.
We used MK-19s, M2 .50cals as the primary weapons systems on all trucks.
The Taliban feared the MK-19 and we would not get attacked at all if we
were rocking the MK-19. My theory is that the enemy feared weapons with
a burst radius as it's much more difficult to find cover from an
explosion.

\end{document}
